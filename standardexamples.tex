%%% -*-LaTeX-*-
%%% standardexamples.tex.orig
%%% Prettyprinted by texpretty lex version 0.02 [21-May-2001]
%%% on Fri Apr 22 08:10:06 2022
%%% for Steve Dunbar (sdunbar@family-desktop)

\documentclass[12pt]{article}

\input{../../../../etc/macros}
\input{../../../../etc/mzlatex_macros}
%% \input{../../../../etc/pdf_macros}

\bibliographystyle{plain}

\begin{document}

\myheader \mytitle

\hr

\sectiontitle{Standard Examples of Markov Chains}

\hr

\usefirefox

\hr

% \visual{Study Tip}{../../../../CommonInformation/Lessons/studytip.png}
% \section*{Study Tip}

% \hr

\visual{Rating}{../../../../CommonInformation/Lessons/rating.png}
\section*{Rating} %one of
% Everyone: contains no mathematics.
% Student: contains scenes of mild algebra or calculus that may require guidance.
Mathematically Mature:  may contain mathematics beyond calculus with
proofs.  % Mathematicians Only: prolonged scenes of intense rigor.

\hr

\visual{Section Starter Question}{../../../../CommonInformation/Lessons/question_mark.png}
\section*{Section Starter Question}

Give an example from elementary probability of a simple Markov chain.

\hr

\visual{Key Concepts}{../../../../CommonInformation/Lessons/keyconcepts.png}
\section*{Key Concepts}

\begin{enumerate}
    \item
        A wide variety of standard examples illustrate Markov chains:
        sums of independent random variables, random walks on finite and
        infinite state spaces, and urn models.
    \item
        Random walks on finite state spaces can have absorbing or
        reflecting boundaries, or even circular boundaries.  Different
        boundary conditions lead to different Markov chains.
    \item
        The physicist P. Ehrenfest proposed an urn model for statistical
        mechanics and kinetic theory.  The motivation is diffusion
        through a membrane.  Two urns labeled \( A \) and \( B \),
        contain a total of \( N \) balls and the state is the number of
        balls in one urn.  Different urn and ball selections lead to
        different Markov chains.
\end{enumerate}

\hr

\visual{Vocabulary}{../../../../CommonInformation/Lessons/vocabulary.png}
\section*{Vocabulary}
\begin{enumerate}
    \item
        In \defn{random walk with boundaries} the state space is the
        finite set of integers \( \set{0, 1, 2, 3 \dots, n} \).  A
        particle at site \( i \) can stay at \( i \) with probability \(
        r_i \) or move to adjacent site \( i+1 \) with probability \( p_i
        \) and to adjacent site \( i-1 \) with probability \( q_i \)
        with \( p_i+r_i+q_i = 1 \).  The transitions at the boundaries \(
        0 \) and \( n \) get special consideration.
    \item
        In a \defn{random walk with absorbing boundaries}, \( r_0 = r_n
        = 1 \) so the particle reaches sites \( 0 \) or \( n \), then it
        stays at that site with probability \( 1 \).
    \item
        In a \defn{random walk with reflecting boundaries} if the
        particle reaches sites \( 0 \) or \( n \), then at the next step
        it is at \( 1 \) or \( n-1 \) respectively with probability \( 1
        \).
    \item
        In a \defn{random walk on a circle} if the particle reaches site
        \( 0 \), then at the next step it is at \( n \) or \( 1 \) with
        probability \( q \) or \( p \) respectively.  If the particle
        reaches site \( n \), then at the next step it is at \( n-1 \)
        or \( 1 \) with probability \( q \) or \( p \) respectively.
    \item
        In \defn{unbiased random walk on the integers}, also called
        symmetric random walk, the state space is the infinite set of
        integers \( \Integers \).  A particle at site \( i \) moves to
        adjacent site \( i+1 \) with probability \( 1/2 \) and to
        adjacent site \( i-1 \) with probability \( 1/2 \).
    \item
        In \defn{biased random walk on the integers} the state space is
        the infinite set of integers \( \Integers \).  A particle at
        site \( i \) moves to adjacent site \( i+1 \) with probability \(
        p \) and to adjacent site \( i-1 \) with probability \( q \)
        with \( p \ne q \) and \( p+q = 1 \).
    \item
        A \defn{graph} is a finite or infinite set of nodes, also called
        vertices, \( \set{v_1, v_2, \dots, v_n} \), and an associated
        collection of edges \( \set{v_{ij}, 1 \le i,j \le n} \).  There
        need not be an edge between any two vertices, but if there is a
        connection between vertices \( v_i \) and \( v_j \), then \( v_j
        \) is in the neighborhood \( N(v_i) \) of \( v_i \).
    \item
        In the more general \defn{random walk on a graph} a particle at
        vertex \( v_i \) will move to a vertex \( v_j \) in the
        neighborhood \( N(v_i) \) with probability \( p_{ij} \) and may
        stay at \( v_i \) with probability \( p_{ii} \), where \( \sum_
        {k \in [\set{i} \union N(v_i)]} p_{ik} = 1 \).
    \item
        In the \defn{Ehrenfest urn model} a ball is selected at random
        from one of two urns with all selections equally likely, and
        moved from the urn it is in to the other urn.  The state from \(
        0 \) to \( N \) at each time is the number of balls in a fixed
        urn.
\end{enumerate}

\hr

\section*{Notation}
\begin{enumerate}
    \item
        \( i, j, l \) -- arbitrary states of the Markov chain
    \item
        \( \xi \) -- a discrete-valued random variable with possible
        values as the nonnegative integers \( \Prob{\xi=i} = p_i \)
    \item
        \( \xi_1, \xi_2, \xi_3, \dots \) -- independent observations of \(
        \xi \)
    \item
        \( X_n \) -- Markov chain with transition probability matrix \(
        P \)
    \item
        successive partial sums \( \eta_n \) of the random variables of
        the independent observations of the \( \xi_i \)
    \item
        \( p_i, q_i, r_i \) -- A particle at site \( i \) can stay at \(
        i \) with probability \( r_i \) or move to adjacent site \( i+1 \)
        with probability \( p_i \) and to adjacent site \( i-1 \) with
        probability \( q_i \) with \( p_i+r_i+q_i = 1 \).
    \item
        \( p, q = 1-p \) -- simple Bernoulli probabilities
    \item
        \( n, t \) -- time steps or periods of the Markov chain
    \item
        \( \pi \) -- stationary distribution for a Markov chain
    \item
        \( k \) -- related to the number of states of the Markov chain (\(
        k + 1 \) states if there is a \( 0 \) state)
\end{enumerate}

\visual{Mathematical Ideas}{../../../../CommonInformation/Lessons/mathematicalideas.png}
\section*{Mathematical Ideas}

\subsection*{State Homogeneous Markov Chains}

\subsubsection*{Simple State Homogeneous Markov Chain}

Let \( \xi \) denote a discrete-valued random variable with possible
values as the nonnegative integers \( \Prob{\xi=i} = p_i \), where \( p_i
\ge 0 \) and \( \sum_{i=1}^{\infty} p_i = 1 \).  Let \( \xi_1, \xi_2,
\xi_3, \dots \) be independent observations of \( \xi \).  Then consider
the process \( X_i \), \( n = 1, 2, 3, \dots \) defined by \( X_n = \xi_n
\).  The transition probability matrix of this state homogeneous Markov
chain is schematically%
\index{Markov chain!state homogeneous}
\[
    P =
    \begin{pmatrix}
        p_1 & p_2 & p_3 & \dots \\
        p_1 & p_2 & p_3 & \dots \\
        p_1 & p_2 & p_3 & \dots \\
        \vdots & \vdots & \vdots & \ddots
    \end{pmatrix}
    .
\] If \( p_i = 0 \) for \( i > k \), then the state space of the Markov
chain is finite, with corresponding \( k \times k \) finite transition
probability matrix.

In this example, make the sensible assumption that \( 0 < p_i < 1 \) for
the set of states, either finite or infinite, so that only states with a
probability of appearing are considered.  The stationary distribution is
\( p_i > 0 \).  Note that \( P^n = P \).  With the assumption, this
Markov chain is
\begin{itemize}
    \item
        ergodic,
    \item
        regular,
    \item
        aperiodic,
    \item
        each state is accessible and all states communicate,
    \item
        recurrent,
    \item
        has no absorbing states.
\end{itemize}
This Markov chain is equivalent to independent samples from the
distribution \( \Prob{\xi=i} = p_i \).

\subsubsection*{Sums of Independent Random Variables as Markov Chains}

Another important Markov chain comes from successive partial sums \(
\eta_n \) of the random variables of the independent observations of the
\( \xi_i \)
\[
    \eta_n = \xi_1 + \xi_2 + \xi_3 + \cdots + \xi_n
\] with \( \eta_0 = 0 \) by definition.%
\index{sums of independent random variables}
\index{Markov chain!sums of independent random variables}
\index{sums of independent random variables!as Markov chain}
The entries of the probability transition matrix are defined by
\begin{align*}
    &\Prob{X_{n+1} = j \given X_n = i} \\
    & = \Prob{\xi_1 + \xi_2 + \xi_3 + \cdots + \xi_n + \xi_{n+1} = j
    \given \xi_1 + \xi_2 + \xi_3 + \cdots + \xi_n = i} \\
    & = \Prob{\xi_{n+1} = j-i}
\end{align*}
assuming the independence of the \( \xi_i \).  Schematically the
probability transition matrix is
\[
    P =
    \begin{pmatrix}
        p_1 & p_2 & p_3 & p_4 & \dots \\
        0 & p_1 & p_2 & p_3 & \dots \\
        0 & 0 & p_1 & p_2 & \dots \\
        \vdots & \vdots & \vdots & \vdots & \ddots
    \end{pmatrix}
    .
\]

If the random variables \( \xi \) assumes both positive and negative
integers, then the possible values of the partial sums \( \eta_n \) will
be contained in the set of all integers.  Then let the state space be \(
\Integers \).  Letting \( \Prob{\xi = i} = p_{i} \), \( i \in \Integers \),
with \( p_i \ge 0 \), \( \sum_{i=-\infty}^{\infty} p_i = 1 \).  Then
schematically the probability transition matrix is
\[
    P =
    \begin{pmatrix}
        \vdots & \vdots & \vdots & \vdots \vdots & \vdots & \vdots &
        \vdots \\
        \cdots & p_{-1} & p_0 & p_1 & p_2 & p_3 & p_4 & \cdots \\
        \cdots & p_{-2} & p_{-1} & p_0 & p_1 & p_2 & p_3 & \cdots \\
        \cdots & p_{-3} & p_{-2} & p_{-1} & p_0 & p_1 & p_2 & \cdots \\
        \vdots & \vdots & \vdots & \vdots \vdots & \vdots & \vdots &
        \vdots
    \end{pmatrix}
    .
\]

This Markov chain is quite general and so special cases will be
considered as examples of the various characteristics of Markov chains.

\subsection*{Random Walks as Markov Chains}

\subsubsection*{Random Walk with Absorbing Boundaries}

In \defn{random walk with boundaries} the state space is the finite set
of integers \( \set{0, 1, 2, 3 \dots, k} \).%
\index{random walk! with boundaries}
A particle at site \( i \) can stay at \( i \) with probability \( r_i \)
or move to adjacent site \( i+1 \) with probability \( p_i \) and to
adjacent site \( i-1 \) with probability \( q_i \) with \( p_i+r_i+q_i =
1 \).  The transitions at the boundaries \( 0 \) and \( k \) get special
consideration.  The transition probability matrix in general is
\[
    P =
    \begin{pmatrix}
        r_0 & p_0 & 0 & 0 & 0 & \cdots & 0 & 0 & 0 \\
        q_1 & r_1 & p_1 & 0 & 0 & \cdots & 0 & 0 & 0 \\
        0 & q_2 & r_2 & p_2 & 0 & \cdots & 0 & 0 & 0 \\
        0 & 0 & q_3 & r_3 & p_3 & \cdots & 0 & 0 & 0 \\
        \vdots & \vdots & \vdots & \vdots & \vdots & \ddots & \vdots
        \vdots & \vdots \\
        0 & 0 & 0 & 0 & 0 & \cdots & 0 & q_k & r_k \\
    \end{pmatrix}
    .
\]

If \( r_0 = r_k = 1 \) the particle reaches sites \( 0 \) or \( k \),
then it stays at that site with probability \( 1 \).  This is a \defn{random
walk with absorbing boundaries}.  In a common%
\index{random walk! with absorbing boundaries}
\index{Markov chain ! random walk with absorbing boundaries}
special case, the transition probabilities do not depend on \( i \) and \(
r_i = 0 \).  The transition probability matrix is
\[
    P =
    \begin{pmatrix}
        1 & 0 & 0 & 0 & 0 & \cdots & 0 & 0 & 0 \\
        q & 0 & p & 0 & 0 & \cdots & 0 & 0 & 0 \\
        0 & q & 0 & p & 0 & \cdots & 0 & 0 & 0 \\
        0 & 0 & q & 0 & p & \cdots & 0 & 0 & 0 \\
        \vdots & \vdots & \vdots & \vdots & \vdots & \ddots & \vdots &
        \vdots & \vdots \\
        0 & 0 & 0 & 0 & 0 & \cdots & 0 & 0 & 1 \\
    \end{pmatrix}
    .
\] This is also the Markov chain for the gambler's ruin.%
\index{gambler's ruin}
\index{Markov chain!gambler's ruin}

This Markov chain is not ergodic because of the absorbing boundaries. In
the case that \( r_i = 0 \), the period of Markov chain is \( 2 \). If \(
r_i = 0 \) the chain is not regular.  All states are accessible from
states \( i \) for \( 0 < i < k \), but because of the absorbing
boundaries, not all states communicate.  The transient states are \( 1,
2, \dots, k-1 \) and the absorbing states are \( 0 \) and \( k \).
Therefore, it does not make sense to consider any stable distributions.

\subsubsection*{Random Walk with Reflecting Boundaries}

In a \defn{random walk with reflecting boundaries} the state space is
the finite set of integers \( \set{0, 1, 2, 3 \dots, k} \).  A particle
at%
\index{random walk! with reflecting boundaries}
\index{Markov chain ! random walk with reflecting boundaries}
site \( i \) with \( 1 < i < k-1 \) moves to adjacent site \( i+1 \)
with probability \( p \) and to adjacent site \( i-1 \) with probability
\( q \) with \( p+q = 1 \).  If the particle reaches sites \( 0 \) or \(
k \), then at the next step it is at \( 1 \) or \( n-1 \) respectively
with probability \( 1 \).  The transition probability matrix is
\[
    P =
    \begin{pmatrix}
        0 & 1 & 0 & 0 & 0 & \cdots & 0 & 0 & 0 \\
        q & 0 & p & 0 & 0 & \cdots & 0 & 0 & 0 \\
        0 & q & 0 & p & 0 & \cdots & 0 & 0 & 0 \\
        \vdots & \vdots & \vdots & \vdots & \vdots & \ddots & \vdots &
        \vdots & \vdots \\
        0 & 0 & 0 & 0 & 0 & \cdots & 0 & 1 & 0 \\
    \end{pmatrix}
    .
\]

For this transition probability matrix \( P \) there is a probability
distribution \( \pi \) such that \( \pi P = \pi \).  However it is not
strictly true that the Markov chain has a stationary distribution
because the chain has period \( 2 \).  However, this can be fixed by
embedding this random walk in a \emph{lazy random walk}%
\index{lazy random walk}
sequence \( X_0, X_1, \dots, X_{n-1}m X_n, \dots \) where given \( X_{n-1}
\), first choose to remain at \( X_{n-1} \) with probability \( \frac{1}
{2} \).  Alternatively, with probability \( \frac{1}{2} \) choose to
move to \( X_n \) selected with probability \( q \) and \( p \) from the
left and right nodes adjacent to \( x_{n-1} \).  This Markov chain is
ergodic.  The chain is periodic with period \( 2 \) so is not regular.
All states are accessible and all states communicate.  This Markov chain
has no transient states and all states are recurrent.  The Markov chain
has no absorbing states.

\subsubsection*{Random Walk on a Circle}

In \defn{random walk on a circle} the state space is the finite set of
integers \( \set{0, 1, 2, 3 \dots, k} \).  A particle at%
\index{random walk! on a circle}
\index{Markov chain ! random walk on a circle}
site \( i \) with \( 1 < i < k-1 \) moves to adjacent site \( i+1 \)
with probability \( p \) and to adjacent site \( i-1 \) with probability
\( q \) with \( p+q = 1 \).  If the particle reaches sites \( 0 \), then
at the next step it is at \( k \) or \( 1 \) with probability \( q \) or
\( p \) respectively.  If the particle reaches sites \( k \), then at
the next step it is at \( k-1 \) or \( 1 \) with probability \( q \) or \(
p \) respectively.  The transition probability matrix is
\[
    P =
    \begin{pmatrix}
        0 & p & 0 & 0 & 0 & \cdots & 0 & 0 & q \\
        q & 0 & p & 0 & 0 & \cdots & 0 & 0 & 0 \\
        0 & q & 0 & p & 0 & \cdots & 0 & 0 & 0 \\
        \vdots & \vdots & \vdots & \vdots & \vdots & \ddots & \vdots &
        \vdots & \vdots \\
        p & 0 & 0 & 0 & 0 & \cdots & 0 & q & 0 \\
    \end{pmatrix}
    .
\] This is also a random walk with state space \( \Integers_{k+1} \).

This Markov chain has a stationary distribution
\[
    (1/(k+1), 1/(k+1),\dots, 1/(k+1))
\] (subject to the embedding into a lazy random walk as in the previous
example.) This Markov chain is ergodic.  The chain is periodic with
period \( 2 \) so it is not regular.  All states are accessible and all
states communicate.  This Markov chain has no transient states and all
states are recurrent.  This Markov chain has no absorbing states.

\subsubsection*{Unbiased Walk on Integers}

In \defn{unbiased random walk on the integers}, also called symmetric%
\index{unbiased random walk on the integers}
\index{random walk! unbiased}
\index{Markov chain ! unbiased walk on integers}
random walk, the state space is the infinite set of integers \(
\Integers \).  A particle at site \( i \) with \( i \) moves to adjacent
site \( i+1 \) with probability \( 1/2 \) and to adjacent site \( i-1 \)
with probability \( 1/2 \).  The transition probability is
\[
    p_{ij} =
    \begin{cases}
        1/2 & j = i-1 \\
        1/2 & j = i+1 \\
        0 & j \ne i-1, i+1 \\
    \end{cases}
    .
\]

Because this Markov chain has an infinite state space, it does not make
sense to consider a stable distribution.  This Markov chain is ergodic.
The chain is periodic with period \( 2 \) so it is not regular.  All
states are accessible and all states communicate.  It is a deep theorem
that the Markov chain has no transient states and all states are
recurrent, see Theorem 7 and the remarks following Theorem 7 in \link{Recurrence}
{http://www.math.unl.edu/~sdunbar1/ProbabilityTheory/Lessons/BernoulliTrials/Recurrence/recurrence.html}.
Therefore, this is an example of an infinite state space Markov chain
that is ergodic and is also recurrent.  This Markov chain has no
absorbing states.

\subsubsection*{Biased Walk on Integers}

In \defn{biased random walk on the integers} the state space is the%
\index{unbiased random walk on the integers}
\index{random walk! biased}
\index{Markov chain ! biased walk on integers}
infinite set of integers \( \Integers \).  A particle at site \( i \)
with \( i \) moves to adjacent site \( i+1 \) with probability \( p \)
and to adjacent site \( i-1 \) with probability \( q \) with \( p \ne q \)
and \( p+q = 1 \).  The transition probability is
\[
    p_{ij} =
    \begin{cases}
        p & j = i-1 \\
        q & j = i+1 \\
        0 & j \ne i-1, i+1 \\
    \end{cases}
    .
\]

Because this Markov chain has an infinite state space, it does not make
sense to consider a stable distribution.  This Markov chain is ergodic.
The chain is periodic with period \( 2 \) so it is not regular.  All
states are accessible and all states communicate.  It is a substantial
theorem that all states are transient states and no states are
recurrent, see Theorem 7 and the remarks following Theorem 7 in \link{Recurrence}
{http://www.math.unl.edu/~sdunbar1/ProbabilityTheory/Lessons/BernoulliTrials/Recurrence/recurrence.html}.
Therefore, this is an example of an infinite state space Markov chain
that is ergodic but is not recurrent.  This shows that ergodicity and
recurrence are different concepts.  This Markov chain has no absorbing
states.

\subsubsection*{Random Walks on Graphs}

Each of these examples of a Markov chain is a specific case of a more
general \defn{random walk on a graph}.  A \defn{graph} is a finite or%
\index{random walk! on graphs}
\index{Markov chain ! random walk on a graph}
infinite set of nodes, also called vertices, \( \set{v_1, v_2, \dots, v_k}
\), and an associated collection of edges \( \set{v_{ij}, 1 \le i,j \le
n} \).  There need not be an edge between two vertices.  If there is a
connection between vertices \( v_i \) and \( v_j \), then \( v_j \) is
in the neighborhood \( N(v_i) \) of \( v_i \).%
\index{graph}
\index{graph!neighborhood}
A particle at vertex \( v_i \) will move to a vertex \( v_j \) in the
neighborhood \( N(v_i) \) with probability \( p_{ij} \) and may stay at \(
v_i \) with probability \( p_{ii} \), where \( \sum_{j \in [\set{i}
\union N(v_i)]} p_{ij} = 1 \).

\begin{example}
    Consider the \( 3 \times 3 \) square lattice graph in Figure~%
    \ref{fig:standardexamples:sqlattice}.  The graph has \( 9 \)
    vertices with \( 10 \) edges between nearest lattice neighbors.  If
    a vertex has \( n \) edges then the probability of moving to a
    neighboring edge or staying at the vertex is \( \frac{1}{n+1} \)
    uniformly. The random walk on this graph is often colorfully
    characterized as a frog hopping among lily pads or a bug leaping
    among plants.  The transition probability matrix for this graph
    random walk is
    \[
        \begin{pmatrix}
            1/3 & 1/3 & 0 & 1/3 & 0 & 0 & 0 & 0 & 0 \\
            1/4 & 1/4 & 1/4 & 0 & 1/4 & 0 & 0 & 0 & 0 \\
            0 & 1/3 & 1/3 & 0 & 0 & 1/3 & 0 & 0 & 0 \\
            1/4 & 0 & 0 & 1/4 & 1/4 & 0 & 1/4 & 0 & 0 \\
            0 & 1/5 & 0 & 1/5 & 1/5 & 1/5 & 0 & 1/5 & 0 \\
            0 & 0 & 1/4 & 0 & 1/4 & 1/4 & 0 & 0 & 1/4 \\
            0 & 0 & 0 & 1/3 & 0 & 0 & 1/3 & 1/3 & 0 \\
            0 & 0 & 0 & 0 & 1/4 & 0 & 1/4 & 1/4 & 1/4 \\
            0 & 0 & 0 & 0 & 0 & 1/3 & 0 & 1/3 & 1/3
        \end{pmatrix}
        .
    \]

    This Markov chain has a stationary distribution
    \[
        \pi = (\frac{1}{11}, \frac{4}{33}, \frac{1}{11}, \frac{4}{33},
        \frac{5}{33}, \frac{4}{33}, \frac{1}{11}, \frac{4}{33}, \frac{1}
        {11}).
    \] This Markov chain is ergodic.  The chain is aperiodic.  All
    states are accessible and all states communicate.  This Markov chain
    has no transient states and all states are recurrent.  This Markov
    chain has no absorbing states.

    \begin{figure}
        \centering
\begin{asy}
size(5inches);

real myfontsize = 12;
real mylineskip = 1.2*myfontsize;
pen mypen = fontsize(myfontsize, mylineskip);
defaultpen(mypen);

real marge=1mm;
pair z1=(0, 2), z2=(1, 2), z3=(2, 2);
pair z4=(0, 1), z5=(1, 1), z6=(2, 1);
pair z7=(0, 0), z8=(1, 0), z9=(2, 0);

transform r=scale(1.0);

object state1=draw("1",ellipse,z1,marge),
state2=draw("2",ellipse,z2,marge),
state3=draw("3",ellipse,z3,marge),
state4=draw("4",ellipse,z4,marge),
state5=draw("5",ellipse,z5,marge),
state6=draw("6",ellipse,z6,marge),
state7=draw("7",ellipse,z7,marge),
state8=draw("8",ellipse,z8,marge),
state9=draw("9",ellipse,z9,marge);

add(new void(picture pic, transform t) {
    draw(pic, point(state1,E,t)--point(state2,W,t));
    draw(pic, point(state1,S,t)--point(state4,N,t));
});

add(new void(picture pic, transform t) {
    draw(pic, point(state2,E,t)--point(state3,W,t));
    draw(pic, point(state2,S,t)--point(state5,N,t));
});

add(new void(picture pic, transform t) {
    draw(pic, point(state3,S,t)--point(state6,N,t));
});

add(new void(picture pic, transform t) {
    draw(pic, point(state4,E,t)--point(state5,W,t));
    draw(pic, point(state4,S,t)--point(state7,N,t));
});

add(new void(picture pic, transform t) {
    draw(pic, point(state5,E,t)--point(state6,W,t));
    draw(pic, point(state5,S,t)--point(state8,N,t));
});

add(new void(picture pic, transform t) {
    draw(pic, point(state5,E,t)--point(state6,W,t));
    draw(pic, point(state5,S,t)--point(state8,N,t));
});

add(new void(picture pic, transform t) {
    draw(pic, point(state6,S,t)--point(state9,N,t));
});

add(new void(picture pic, transform t) {
    draw(pic, point(state7,E,t)--point(state8,W,t));
});

add(new void(picture pic, transform t) {
    draw(pic, point(state8,E,t)--point(state9,W,t));
});
\end{asy}
        \caption{A $3 \times 3$ square lattice graph with uniform
        transition probabilities.}%
        \label{fig:standardexamples:lattice3x3}
    \end{figure}

    The stationary distribution \( \pi \) is unique and the chain
    converges to \( \pi \) as \( n \to \infty \).  Later sections will
    use this Markov chain as an example about quantitative rates of
    convergence, that is, how large \( n \) must be to make the chain
    sufficiently close to \( \pi \).
\end{example}

\subsection*{Urn Models}

\subsubsection*{Ehrenfest Urn Model}

The physicist P. Ehrenfest proposed the following model for statistical
mechanics and kinetic theory.  The motivation is diffusion through a
membrane.  Two urns labeled \( A \) and \( B \) contain a total of \( k \)
balls.  In the \defn{Ehrenfest urn model} a%
\index{Ehrenfest urn model}
\index{Markov chain ! Ehrenfest urn model}
\index{urn model}
ball is selected at random with all selections equally likely, and moved
from the urn it is in to the other urn.  The state at each time is the
number of balls in the urn \( A \), from \( 0 \) to \( k \).  Then the
transition probability matrix is
\[
    P =
    \begin{pmatrix}
        0 & 1 & 0 & 0 & \cdots & 0 & 0 \\
        \frac{1}{k} & 0 & 1-\frac{1}{k} & 0 & \cdots & 0 & 0 \\
        0 & \frac{2}{k} & 0 & 1-\frac{2}{k} & \cdots & 0 & 0 \\
        \vdots & \vdots & \vdots & \vdots & \ddots& \vdots & \vdots \\
        0 & 0 & 0 & 0 & \cdots & 0 & 1/k \\
        0 & 0 & 0 & 0 & \cdots & 1 & 0 \\
    \end{pmatrix}
    .
\] The balls fluctuate between the two containers with a drift from the
one with the larger number of balls to the one with the smaller numbers.

A stationary distribution for this Markov chain has entry \( \pi_i =
\binom{k}{i}/2^k \) (subject to the embedding into a lazy Markov chain
as in the previous examples.).  (See the exercises.) This is the
binomial distribution on \( k \), so that for large \( k \), this can be
approximated with the normal distribution.  This conclusion is plausible
given the physical origin of the Markov chain as a model for diffusion.
This Markov chain is ergodic.  All states are accessible and all states
communicate.  The chain is periodic with period \( 2 \) so it is not
regular.  All states are recurrent with no transient states.  This
Markov chain has no absorbing states.

\subsubsection*{An Alternate Ehrenfest Urn Model}
% Karlin and Taylor, Chapter 2, Problem 2 (a), page 74

Two urns labeled \( A \) and \( B \), contain a total of \( k \) balls.
A ball is selected at random with all selections equally likely.  Then
an urn is selected, urn \( A \) with probability \( p \) and urn \( B \)
with probability \( q = 1-p \) and the ball is moved to that urn.  The
state at each time is the number of balls in the urn \( A \), from \( 0 \)
to \( N \).  Then the transition probability matrix is
\[
    P =
    \begin{pmatrix}
        q & p & 0 & 0 & \cdots & 0 & 0 \\
        \frac{1}{k} q & \frac{1}{k}p + \left( 1-\frac{1}{k} \right) q &\left
        (1-\frac{1}{k} \right) p & 0 & \cdots & 0 & 0 \\
        0 & \frac{2}{k} q &\frac{2}{k} p + \left( 1-\frac{2}{k} \right)
        q &\left( 1-\frac{2}{k} \right) p & \ dots & 0 & 0 \\
        0 & 0 & \frac{3}{k} q & \frac{3}{k} p + \left( 1-\frac{3}{k}
        \right) q & \cdots & 0 & 0 \\
        \vdots & \vdots & \vdots & \vdots & \ddots & \vdots & \vdots \\
        0 & 0 & 0 & 0 & \cdots & q & p \\
    \end{pmatrix}
    .
\]

A stationary distribution for this Markov chain exists, but depends on \(
p \), \( i \), and \( k \) in a more complicated relationship than in
the previous example.  This Markov chain is ergodic.  All states are
accessible and all states communicate.  The chain is periodic with
period \( 1 \) so it is regular.  All states are recurrent and there are
no transient states.  This Markov chain has no absorbing states.

\subsubsection*{Another Ehrenfest Urn Model}
% Karlin and Taylor, Chapter 2, Problem 2 (b), page 74

Two urns labeled \( A \) and \( B \) contain a total of \( k \) balls.
At time \( t \) there \( j \) balls in urn \( A \).  At time \( t + 1 \)
an urn is selected, urn \( A \) with probability \( j/k \) and urn \( B \)
with probability \( (k-j)/k \).  Then independently a ball is selected
from urn \( A \) with probability \( p \) or from urn \( B \) with
probability \( q = 1-p \) and placed in the previously selected urn.
The state at each time is the number of balls in the urn \( A \), from \(
0 \) to \( k \).  Then the transition probability matrix is
\[
    P =
    \begin{pmatrix}
        1 & 0 & 0 & 0 & \cdots & 0 & 0 \\
        \frac{k-1}{k} p & \frac{k-1}{k} q + \left( \frac{1}{k} \right) p
        & \left( \frac{1}{k} \right) q & 0 & \cdots & 0 & 0 \\
        0 & \frac{k-2}{k} p & \left( 1-\frac{2}{k}\right) q + \frac{2}{k}
        p & \left( \frac{2}{k} \right) p & \cdots & 0 & 0 \\
        0 & 0 & \frac{k-3}{k} p & \left( \frac{k-3}{k}\right) q + \frac{3}
        {k} p & \cdots & 0 & 0 \\
        \vdots & \vdots & \vdots & \vdots & \ddots & \vdots & \vdots \\
        0 & 0 & 0 & 0 & \cdots & 0 & 1 \\
    \end{pmatrix}
    .
\]

This Markov chain is not ergodic.  States \( 1 \) through \( k-1 \)
communicate while states \( 0 \) and \( k \) are absorbing.  States \( 1
\) to \( k-1 \) are transient states.  The chain is periodic with period
\( 1 \).

\subsubsection*{Yet Another Generalized Ehrenfest Urn Model}
% Karlin and Taylor, Chapter 2, Problem 2 (c), page 74

Two urns labeled \( A \) and \( B \), contain a total of \( k \) balls.
At time \( t \) there \( k \) balls in urn \( A \).  At time \( t + 1 \)
an urn is selected, urn \( A \) with probability \( k/k \) and urn \( B \)
with probability \( (k-j)/k \).  Then independently a ball is selected
from urn \( A \) with probability \( j/k \) or from urn \( B \) with
probability \( (k-j)/k \) and placed in the previously selected urn. The
state at each time is the number of balls in the urn \( A \), from \( 0 \)
to \( k \).  Then the transition probability matrix is
\[
    P =
    \begin{pmatrix}
        1 & 0 & 0 & 0 & \cdots & 0 & 0 \\
        \frac{1 \cdot(k-1)}{k^2} & \frac{1^2}{k^2} +\frac{(k-1)^2}{k^2}
        & \frac{(k-1) \cdot 2 }{k} & 0 & \cdots & 0 & 0 \\
        0 & \frac{2 \cdot (k-2)}{k} & \frac{2^2}{k^{2}} + \frac{(k-2)^2}
        {k^2} & \frac{(k-2) \cdot 2}{k^{2}} & \cdots & 0 & 0 \\
        0 & 0 & \frac{3 \cdot (k-3)}{k^2} & \frac{3^2}{k^{2}} + \frac{(k-3)^2}
        {k^2} & \cdots & 0 & 0 \\
        \vdots & \vdots & \vdots & \vdots & \ddots & \vdots & \vdots \\
        0 & 0 & 0 & 0 & \cdots & 0 & 1 \\
    \end{pmatrix}
    .
\]

This Markov chain is not ergodic.  States \( 1 \) through \( k-1 \)
communicate while states \( 0 \) and \( k \) are absorbing.  States \( 1
\) to \( k-1 \) are transient states.  The chain is periodic with period
\( 1 \).

\visual{Section Starter Question}{../../../../CommonInformation/Lessons/question_mark.png}
\section*{Section Ending Answer}

A standard example from elementary probability theory that is also a
Markov chain is the sequence of partial sums of a sequence of
independent identically distributed random variables.  This has the
natural interpretation as a random walk, also a simple example of a
Markov chain.

\subsection*{Sources} This section is adapted from:  \booktitle{A First
Course in Stochastic Processes}, by S. Karlin and H. Taylor, Chapter 2,
pages 45--80,
\cite{karlin75}; with more information and ideas from \booktitle{Problems
in Probability Theory, Mathematical Statistics and Theory of Random
Functions} by A. A. Svesnikov, Chapter VIII, pages 231--274,
\cite{svesnikov68}

\hr

\visual{Algorithms, Scripts, Simulations}{../../../../CommonInformation/Lessons/computer.png}
\section*{Algorithms, Scripts, Simulations}

\subsection*{Algorithm}

\begin{algorithm}[H]
    \DontPrintSemicolon
    \KwData{State names and probability transition matrix}
    \KwResult{Information about a simple Markov chain}
    \BlankLine
    \emph{Initialization and sample paths}\;
    Load Markov chain library\;
    Set state names, set transition probability matrix, set start
    state\;
    Set an example length and create a sample path of example length\;
    Create a second sample path of example length\;
    \BlankLine
    \emph{Simulation of stationary distribution and comparison to theoretical}\;
    Set a long path length, and a transient time\;
    Create a long sample path\;
    Slice the long sample path from the transient time to the end\;
    In the slice count the appearance of each state\;
    Store in an empirical array\;
    Compute the theoretical stable array\;

    \KwRet{Stable distribution and theoretical stable distribution}
    \caption{Markov chain simulation.}
\end{algorithm}
\subsection*{Scripts}


\begin{description}

% \item[Geogebra] 

% \link{  .ggb}{GeoGebra applet}

\item[R] 

\link{http://www.math.unl.edu/~sdunbar1/Probability/Lessons/MarkovChains/StandardExamples/absorbingbdy.R}{R script for
  Random Walk with Absorbing Boundaries.}

\begin{lstlisting}[language=R]
library(markovchain)

N <- 7
stateNames <- as.character( 0:N )
transMatrix <- matrix(0, N+1,N+1)
transMatrix[1,1] <- 1
transMatrix[N+1, N+1] <- 1
for (row in 2:N) {  #row 2 is state 1, row 7 is state 6
    transMatrix[row, row-1] <- 1/2
    transMatrix[row, row+1] <- 1/2
}
startState <- "3"

absorbingbdy <- new("markovchain", transitionMatrix=transMatrix,
                 states=stateNames, name="AbsorbingBdy")
print(absorbingbdy)

print( summary(absorbingbdy) )

pathLength <- 10 
path <- rmarkovchain(n=pathLength,
                            object = absorbingbdy, t0 = startState)
cat("A sample starting from ", startState, path, "\n")
path <- rmarkovchain(n=pathLength,
                            object = absorbingbdy, t0 = startState)
cat("Another sample starting from ", startState, path, "\n")

if ( length( transientStates(absorbingbdy) ) == 0 ) {
    largeN <- 1000; startStable <- 100
    path <- rmarkovchain(n=largeN,
                            object = absorbingbdy, t0 = startState)
    stablePattern <-  path[startStable:largeN]
    empiricalStable <- rep(0, N+1)
    for (i in seq_along(stateNames)) { 
        empiricalStable[i] <-  mean( stablePattern == (i-1) )
    }
    theorStable <- steadyStates(absorbingbdy)
    cat("Empirical stable distribution: ",  empiricalStable, "\n")
    cat("Theoretical stable distribution: ", theorStable, "\n")
}
\end{lstlisting}  

\link{http://www.math.unl.edu/~sdunbar1/Probability/Lessons/MarkovChains/StandardExamples/reflectingbdy.R}{R script for
  Random Walk with Reflecting Boundaries.}

\begin{lstlisting}[language=R]
library(markovchain)

N <- 7
stateNames <- as.character( 0:N )
transMatrix <- matrix(0, N+1,N+1)
transMatrix[1,2] <- 1
transMatrix[N+1, N] <- 1
for (row in 2:N) {  #row 2 is state 1, row 7 is state 6
    transMatrix[row, row-1] <- 1/2
    transMatrix[row, row+1] <- 1/2
}
startState <- "3"

reflectingbdy <- new("markovchain", transitionMatrix=transMatrix,
                 states=stateNames, name="ReflectingBdy")
print(reflectingbdy)

print( summary(reflectingbdy) )

pathLength <- 10 
path <- rmarkovchain(n=pathLength,
                            object = reflectingbdy, t0 = startState)
cat("A sample starting from ", startState, path, "\n")
path <- rmarkovchain(n=pathLength,
                            object = reflectingbdy, t0 = startState)
cat("Another sample starting from ", startState, path, "\n")

if ( length( transientStates( reflectingbdy) ) == 0 ) {
    largeN <- 1000; startStable <- 100
    path <- rmarkovchain(n=largeN,
                         object = reflectingbdy, t0 = startState)
    stablePattern <-  path[startStable:largeN]
    empiricalStable <- rep(0, N+1)
    for (i in seq_along(stateNames)) { 
        empiricalStable[i] <-  mean( stablePattern == (i-1) )
    }
    theorStable <- steadyStates(reflectingbdy)
    cat("Empirical stable distribution: ",  empiricalStable, "\n")
    cat("Theoretical stable distribution: ", theorStable, "\n")
}
\end{lstlisting}

\link{http://www.math.unl.edu/~sdunbar1/Probability/Lessons/MarkovChains/StandardExamples/walkcircle.R}{R script for
  Random Walk on a Circle.}

\begin{lstlisting}[language=R]
library(markovchain)

N <- 7
stateNames <- as.character( 0:N )
transMatrix <- matrix(0, N+1,N+1)
transMatrix[1,2] <- 1/2; transMatrix[1,N+1] <- 1/2
transMatrix[N+1, N] <- 1/2; transMatrix[N+1,1] <- 1/2
for (row in 2:N) {  #row 2 is state 1, row 7 is state 6
    transMatrix[row, row-1] <- 1/2
    transMatrix[row, row+1] <- 1/2
}
startState <- "3"

walkcircle <- new("markovchain", transitionMatrix=transMatrix,
                 states=stateNames, name="Walkcircle")
print(walkcircle)

print( summary(walkcircle) )

pathLength <- 10 
path <- rmarkovchain(n=pathLength,
                            object = walkcircle, t0 = startState)
cat("A sample starting from ", startState, path, "\n")
path <- rmarkovchain(n=pathLength,
                            object = walkcircle, t0 = startState)
cat("Another sample starting from ", startState, path, "\n")

if ( length( transientStates(walkcircle) ) == 0 ) {
    largeN <- 1000; startStable <- 100
    path <- rmarkovchain(n=largeN,
                         object = walkcircle, t0 = startState)
    stablePattern <-  path[startStable:largeN]
    empiricalStable <- rep(0, N+1)
    for (i in seq_along(stateNames)) { 
        empiricalStable[i] <-  mean( stablePattern == (i-1) )
    }
    theorStable <- steadyStates(walkcircle)
    cat("Empirical stable distribution: ",  empiricalStable, "\n")
    cat("Theoretical stable distribution: ", theorStable, "\n")
}
\end{lstlisting}


\link{http://www.math.unl.edu/~sdunbar1/Probability/Lessons/MarkovChains/StandardExamples/ehrenfest1.R}{R script for
   Ehrenfest Urn Model.}

\begin{lstlisting}[language=R]
library(markovchain)

N <- 7
stateNames <- as.character( 0:N )
transMatrix <- matrix(0, N+1,N+1)
transMatrix[1,2] <- 1
transMatrix[N+1, N] <- 1
for (row in 2:N) {  #row 2 is state 1, row 7 is state 6
    transMatrix[row, row-1] <- (row-1)/N
    transMatrix[row,row+1] <- (N-(row-1))/N
}
startState <- "3"

ehrenfest1 <- new("markovchain", transitionMatrix=transMatrix,
                 states=stateNames, name="Ehrenfest1")
print(ehrenfest1)

print( summary(ehrenfest1) )

pathLength <- 10 
path <- rmarkovchain(n=pathLength,
                            object = ehrenfest1, t0 = startState)
cat("A sample starting from ", startState, path, "\n")
path <- rmarkovchain(n=pathLength,
                            object = ehrenfest1, t0 = startState)
cat("Another sample starting from ", startState, path, "\n")

if ( length( transientStates(ehrenfest1) ) == 0 ) {
    largeN <- 1000; startStable <- 100
    path <- rmarkovchain(n=largeN,
                         object = ehrenfest1, t0 = startState)
    stablePattern <-  path[startStable:largeN]
    empiricalStable <- rep(0, N+1)
    for (i in seq_along(stateNames)) { 
        empiricalStable[i] <-  mean( stablePattern == stateNames[i] )
    }
    theorStable <- steadyStates(ehrenfest1)
    cat("Empirical stable distribution: ",  empiricalStable, "\n")
    cat("Theoretical stable distribution: ", theorStable, "\n")
}

  
\end{lstlisting}

\link{http://www.math.unl.edu/~sdunbar1/Probability/Lessons/MarkovChains/StandardExamples/ehrenfest2.R}{R script for
  alternative Ehrenfest Urn Model.}

\begin{lstlisting}[language=R]
library(markovchain)

N <- 7; p <- 1/2; q <- 1 - p;
stateNames <- as.character( 0:N )
transMatrix <- matrix(0, N+1,N+1)
transMatrix[1,1] <- q
transMatrix[1,2] <- p
transMatrix[N+1, N  ] <- q
transMatrix[N+1, N+1] <- p
for (row in 2:N) {
    transMatrix[row, row-1] <- ((N-(row-1))/N)*p
    transMatrix[row,row] <- ((N-(row-1))/N)*q + ((row-1)/N)*p
    transMatrix[row,row+1] <- ((row-1)/N)*q
}
startState <- "3"

ehrenfest2 <- new("markovchain", transitionMatrix=transMatrix,
                 states=stateNames, name="Ehrenfest2")
print(ehrenfest2)

print( summary(ehrenfest2) )

pathLength <- 10 
path <- rmarkovchain(n=pathLength,
                            object = ehrenfest2, t0 = startState)
cat("A sample starting from startState ", startState, path, "\n")
path <- rmarkovchain(n=pathLength,
                            object = ehrenfest2, t0 = startState)
cat("Another sample starting from ", startState, path, "\n")

if ( length( transientStates(ehrenfest2) ) == 0 ) {
    largeN <- 1000; startStable <- 100
    path <- rmarkovchain(n=largeN,
                         object = ehrenfest2, t0 = startState)
    stablePattern <-  path[startStable:largeN]
    empiricalStable <- rep(0, N+1)
    for (i in stateNames) { 
        empiricalStable[i] <-  mean( stablePattern == stateNames[i] )
    }
    theorStable <- steadyStates(ehrenfest2)
    cat("Empirical stable distribution: ",  empiricalStable, "\n")
    cat("Theoretical stable distribution: ", theorStable, "\n")
}


  
\end{lstlisting}

\link{http://www.math.unl.edu/~sdunbar1/Probability/Lessons/MarkovChains/StandardExamples/ehrenfest3.R}{R script for
  another alternative Ehrenfest Urn Model.}

\begin{lstlisting}[language=R]
library(markovchain)

N <- 7; p <- 1/2; q <- 1 - p;
stateNames <- as.character( 0:N )
transMatrix <- matrix(0, N+1,N+1)
transMatrix[1,1] <- 1
transMatrix[N+1, N+1] <- 1
for (row in 2:N) {
    transMatrix[row, row-1] <- ((N-(row-1))/N)*p
    transMatrix[row,row] <- ((N-(row-1))/N)*q + ((row-1)/N)*p
    transMatrix[row,row+1] <- ((row-1)/N)*q
}
startState <- "3"

ehrenfest3 <- new("markovchain", transitionMatrix=transMatrix,
                 states=stateNames, name="Ehrenfest3")
print(ehrenfest3)

print( summary(ehrenfest3) )

pathLength <- 10 
path <- rmarkovchain(n=pathLength,
                            object = ehrenfest3, t0 = startState)
cat("A sample starting from ", startState, path, "\n")
path <- rmarkovchain(n=pathLength,
                            object = ehrenfest3, t0 = startState)
cat("Another sample starting from ", startState, path, "\n")

if ( length( transientStates(ehrenfest3) ) == 0 ) {
    largeN <- 1000; startStable <- 100
    path <- rmarkovchain(n=largeN,
                         object = ehrenfest3, t0 = startState)
    stablePattern <-  path[startStable:largeN]
    empiricalStable <- rep(0, N+1)
    for (i in stateNames) { 
        empiricalStable[i] <-  mean( stablePattern == i )
    }
    theorStable <- steadyStates(ehrenfest3)
    cat("Empirical stable distribution: ",  empiricalStable, "\n")
    cat("Theoretical stable distribution: ", theorStable, "\n")
}


  
\end{lstlisting}

\link{http://www.math.unl.edu/~sdunbar1/Probability/Lessons/MarkovChains/StandardExamples/ehrenfest4.R}{R script for
   yet another alternative Ehrenfest Urn Model.}

\begin{lstlisting}[language=R]
library(markovchain)

N <- 7
stateNames <- as.character( 0:N )
transMatrix <- matrix(0, N+1,N+1)
transMatrix[1,1] <- 1
transMatrix[N+1, N+1] <- 1
for (row in 2:N) {  #row 2 is state 1, row 7 is state 6
    transMatrix[row, row-1] <- (row-1)*(N-(row-1))/N^2
    transMatrix[row,row] <- (row-1)^2/N^2 + (N-(row-1))^2/N^2
    transMatrix[row,row+1] <- (row-1)*(N-(row-1))/N^2
}
startState <- "3"

ehrenfest4 <- new("markovchain", transitionMatrix=transMatrix,
                 states=stateNames, name="Ehrenfest4")
print(ehrenfest4)

print( summary(ehrenfest4) )

pathLength <- 10 
path <- rmarkovchain(n=pathLength,
                     object = ehrenfest4, t0 = startState)
cat("A sample starting from ", startState, ":", path, "\n")
path <- rmarkovchain(n=pathLength,
                     object = ehrenfest4, t0 = startState)
cat("Another sample starting from ", startState, ":", path, "\n")

if ( length( transientStates(ehrenfest4) ) == 0 ) {
    largeN <- 1000; startStable <- 100
    path <- rmarkovchain(n=largeN,
                         object = ehrenfest4, t0 = startState)
    stablePattern <-  path[startStable:largeN]
    empiricalStable <- rep(0, N+1)
    for (i in seq_along(stateNames)) { 
        empiricalStable[i] <-  mean( stablePattern == stateNames[i] )
    }
    theorStable <- steadyStates(ehrenfest4)
    cat("Empirical long-time distribution: ",  empiricalStable, "\n")
    cat("Theoretical stable distribution: ", theorStable, "\n")
}


  
\end{lstlisting}

% \item[Octave]

% \link{http://www.math.unl.edu/~sdunbar1/    .m}{Octave script for .}

% \begin{lstlisting}[language=Octave]

% \end{lstlisting}

% \item[Perl] 

% \link{http://www.math.unl.edu/~sdunbar1/    .pl}{Perl PDL script for .}

% \begin{lstlisting}[language=Perl]

% \end{lstlisting}

% \item[SciPy] 

% \link{http://www.math.unl.edu/~sdunbar1/    .py}{Scientific Python script for .}

% \begin{lstlisting}[language=Python]

% \end{lstlisting}

\end{description}



%%% Local Variables:
%%% mode: latex
%%% TeX-master: t
%%% End:


\hr

\visual{Problems to Work}{../../../../CommonInformation/Lessons/solveproblems.png}
\section*{Problems to Work for Understanding}

\renewcommand{\theexerciseseries}{}
\renewcommand{\theexercise}{\arabic{exercise}}

\begin{exercise}
    For each of the urn models, express the transition probability \( P_
    {ij} \) of going to state \( j \) from state \( i \) in a general
    form, using the cases construction.
\end{exercise}
\begin{solution}
    \begin{enumerate}[label=(\alph*)]
    \item
        For the first Ehrenfest model
        \[
            P_{ij} =
            \begin{cases}
                i/k & j = i-1 \\
                (k-i)/k & j = i+1 \\
                0 & \text{ otherwise }
            \end{cases}
            .
        \]
    \item
        For the second Ehrenfest model
        \[
            P_{ij} =
            \begin{cases}
                \frac{i}{k} q & j = i-1 \\
                \frac{i}{k} p + \frac{k-i}{k} q & j = i \\
                \frac{k-i}{k} p & j = i+1 \\
                0 & \text{ otherwise }
            \end{cases}
            .
        \]
    \item
        For the third Ehrenfest model
        \[
            P_{ij} =
            \begin{cases}
                \frac{k-i}{k} p & j = i-1 \\
                \frac{i}{k} p + \frac{k-i}{k} q & j = i \\
                \frac{i}{k} q & j = i+1 \\
                0 & \text{ otherwise }
            \end{cases}
            .
        \]
    \item
        For the fourth Ehrenfest model
        \[
            P_{ij} =
            \begin{cases}
                \frac{i(k-i)}{k^2} & j = i-1 \\
                \frac{i^2}{k^2} + \frac{(k-i)^2}{k^2} & j = i \\
                \frac{i(k-i)}{k^2} & j = i+1 \\
                0 & \text{ otherwise }
            \end{cases}
            .
        \]
\end{enumerate}
\end{solution}

\begin{exercise}
    Prove that the stationary distribution for the first Ehrenfest Urn
    Model has entry \( \pi_i = \binom{k}{i}/2^k \).
\end{exercise}

\begin{solution}
    Entry \( i \) must satisfy
    \[
        \binom{k}{i-1}\left( \frac{1}{2} \right)^k \cdot \frac{k-i}{k} +
        \binom{k}{i+1}\left( \frac{1}{2} \right)^k \cdot \frac{i+1}{k} =
        \binom{k}{i}\left( \frac{1}{2} \right)^k
    \] or more simply
    \[
        \binom{k}{i-1} \cdot \frac{k-i+1}{k} + \binom{k}{i+1} \cdot
        \frac{i+1}{k} = \binom{k}{i}.
    \] This reduces to
    \[
        \binom{k-1}{i-1} + \binom{k-1}{i+1} = \binom{k}{i}.
    \] which is the basic binomial (Pascal triangle) identity.
\end{solution}

\begin{exercise}
    For the third and fourth Ehrenfest urn models with \( k = 7 \) and \(
    p = q = 1/2 \) which have absorbing states \( 0 \) and \( k \), find
    the absorption probabilities and the expected waiting times to
    absorption from each transient state.
\end{exercise}
\begin{solution}
    Using the probability transition matrix from \texttt{ehrnefest3.R}
    and using the \texttt{markovchain} package in R\@:
\begin{verbatim}
R> absorptionProbabilities(ehrenfest3)
         0        7
1 0.984375 0.015625
2 0.890625 0.109375
3 0.656250 0.343750
4 0.343750 0.656250
5 0.109375 0.890625
6 0.015625 0.984375
R> meanAbsorptionTime(ehrenfest3)
        1         2         3         4         5         6 
 3.033333  7.233333 10.733333 10.733333  7.233333  3.033333 
\end{verbatim}

    Using the probability transition matrix from \texttt{ehrnefest4.R}
    and using the \texttt{markovchain} package in R\@:
\begin{verbatim}
R> absorptionProbabilities(ehrenfest4)
          0         7
1 0.8571429 0.1428571
2 0.7142857 0.2857143
3 0.5714286 0.4285714
4 0.4285714 0.5714286
5 0.2857143 0.7142857
6 0.1428571 0.8571429
R> meanAbsorptionTime(ehrenfest4)
       1        2        3        4        5        6 
17.15000 26.13333 30.21667 30.21667 26.13333 17.15000 
\end{verbatim}
\end{solution}

\begin{exercise}
    A certain system depends on a single machine that breaks down on any
    given day with probability \( p \).  It takes two days to restore
    the failed machine to normal service.  Form a Markov chain by taking
    as states the pairs \( (x,y) \) where \( x \) is \( 1 \) if the
    machine is in operating condition at the end of a day, \( 0 \)
    otherwise and \( y \) is the number of days, \( 0 \), \( 1 \), or \(
    2 \) the machine has been broken.  Compute the long-term system
    availability if \( p = 0.01, 0.02, 0.05 \) and \( 0.10 \).
\end{exercise}
\begin{solution}
    \[
        P =
        \begin{pmatrix}
            q & p & 0 \\
            0 & 0 & 1 \\
            1 & 0 & 0
        \end{pmatrix}
    \] and
    \[
        (\pi_1, \pi_2,\pi_3) = \left( \frac{1}{1+2p}, \frac{p}{1+2p},
        \frac{p}{1 + 2p} \right).
    \]
\end{solution}

\begin{exercise}
    Given \( m \) white balls and \( m \) black balls which are mixed
    thoroughly and then equally distributed in two urns.  From each urn
    one ball is randomly drawn and placed in the other urn.  Find the
    transition matrix and the probability that after a very large number
    of turns the first urn will have \( k \) white balls.
\end{exercise}
\begin{solution}
    % Sveshnikov, problem 38.17, p.244. This is a generalization of
    % Ross, Probability Models, 3rd edition, page 177, problem 1
    Let the state be the number \( j \) of white balls.  \( p_{jj} = 2j(m-j)/m^2
    \), \( p_{j,j+1} = (m-j)^2/m^2 \), \( p_{j,j-1} = j^2/m^2 \).  After
    a large number of turns \( p_k^{\infty} = \binom{m}{k} p_0^{\infty} \),
    where \( 1/p_0^{\infty} =\sum_{i=0}^m \left( \binom{m}{i} \right)^2
    = \binom{2m}{m} \) and \( p_{ij}^{\infty} = \left( \binom{m}{i}
    \right)^2/ \binom{2m}{m} \).
\end{solution}
\begin{exercise}
    Unknown to public health officials, a person with a highly
    contagious disease enters the population.  During each time period
    the person infects a new person with probability \( p \) or his
    symptoms appear and public health officials discover him, occurring
    with probability \( 1-p \) and quarantine him so that he is unable
    to infect additional people.  Compute the probability distribution
    of the number of infected but undiscovered people in the population
    at the time of first discovery of a carrier.  Assume each infected
    person behaves like the first.
\end{exercise}
\begin{solution}
    The state space is the set of nonnegative integers.  At time period \(
    t \), suppose there are \( i \) infected individuals not in
    quarantine.  Suppose that \( j \) ( \( 0 \le j \le i \)) of these
    contagious individuals infect another \( j \) individuals and that \(
    i-j \) are discovered and moved to quarantine.  At the next time
    period there are a net number \( i + j - (i-j) = 2j \) infectives in
    the population with probability \( \binom{i}{j}p^j(1-p)^j \).  This
    is a discrete time Markov chain on an infinite state space with
    transition probability \( p_{i\ell} = \binom{i}{\ell/2}p^{\ell/2}(1-p)^
    {i-(\ell/2)} \) for \( i \ge 0 \) and \( 0 \le \ell \le 2i \) and \(
    \ell \) even.

    The probability that by period \( 1 \) the one carrier has not been
    discovered is \( p \), by period \( 2 \) there are \( 2 \) carriers
    and the probability that each is not yet discovered is \( p^{2} \),
    by period three there are \( 4 \) carries with no discoveries with
    probability \( p^{4} \), and by period \( n \) there are \( 2^{n-1} \)
    carriers with no discoveries \( p^{2^{n-1}} \).
\end{solution}

\hr

\visual{Books}{../../../../CommonInformation/Lessons/books.png}
\section*{Reading Suggestion:}

\bibliography{../../../../CommonInformation/bibliography}

%   \begin{enumerate}
%     \item
%     \item
%     \item
%   \end{enumerate}

\hr

\visual{Links}{../../../../CommonInformation/Lessons/chainlink.png}
\section*{Outside Readings and Links:}
\begin{enumerate}
    \item
    \item
    \item
    \item
\end{enumerate}

\section*{\solutionsname} \loadSolutions

\hr

\mydisclaim \myfooter

Last modified:  \flastmod

\end{document}

File name :  standardexamples.tex Number of characters :  31206 Number
of words :  3657 Percent of complex words :  18.81 Average syllables per
word :  1.7555 Number of sentences :  153 Average words per sentence :
23.9020 Number of text lines :  656 Number of blank lines :  129 Number
of paragraphs :  119

READABILITY INDICES

Fog :  17.0861 Flesch :  34.0561 Flesch-Kincaid :  14.4471

%%% Local Variables:
%%% mode: latex
%%% TeX-master: t
%%% End:
