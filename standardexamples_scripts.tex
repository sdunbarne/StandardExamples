
\begin{description}

% \item[Geogebra] 

% \link{  .ggb}{GeoGebra applet}

\item[R] 

\link{http://www.math.unl.edu/~sdunbar1/Probability/Lessons/MarkovChains/StandardExamples/absorbingbdy.R}{R script for
  Random Walk with Absorbing Boundaries.}

\begin{lstlisting}[language=R]
library(markovchain)

N <- 7
stateNames <- as.character( 0:N )
transMatrix <- matrix(0, N+1,N+1)
transMatrix[1,1] <- 1
transMatrix[N+1, N+1] <- 1
for (row in 2:N) {  #row 2 is state 1, row 7 is state 6
    transMatrix[row, row-1] <- 1/2
    transMatrix[row, row+1] <- 1/2
}
startState <- "3"

absorbingbdy <- new("markovchain", transitionMatrix=transMatrix,
                 states=stateNames, name="AbsorbingBdy")
print(absorbingbdy)

print( summary(absorbingbdy) )

pathLength <- 10 
path <- rmarkovchain(n=pathLength,
                            object = absorbingbdy, t0 = startState)
cat("A sample starting from ", startState, path, "\n")
path <- rmarkovchain(n=pathLength,
                            object = absorbingbdy, t0 = startState)
cat("Another sample starting from ", startState, path, "\n")

if ( length( transientStates(absorbingbdy) ) == 0 ) {
    largeN <- 1000; startStable <- 100
    path <- rmarkovchain(n=largeN,
                            object = absorbingbdy, t0 = startState)
    stablePattern <-  path[startStable:largeN]
    empiricalStable <- rep(0, N+1)
    for (i in seq_along(stateNames)) { 
        empiricalStable[i] <-  mean( stablePattern == (i-1) )
    }
    theorStable <- steadyStates(absorbingbdy)
    cat("Empirical stable distribution: ",  empiricalStable, "\n")
    cat("Theoretical stable distribution: ", theorStable, "\n")
}
\end{lstlisting}  

\link{http://www.math.unl.edu/~sdunbar1/Probability/Lessons/MarkovChains/StandardExamples/reflectingbdy.R}{R script for
  Random Walk with Reflecting Boundaries.}

\begin{lstlisting}[language=R]
library(markovchain)

N <- 7
stateNames <- as.character( 0:N )
transMatrix <- matrix(0, N+1,N+1)
transMatrix[1,2] <- 1
transMatrix[N+1, N] <- 1
for (row in 2:N) {  #row 2 is state 1, row 7 is state 6
    transMatrix[row, row-1] <- 1/2
    transMatrix[row, row+1] <- 1/2
}
startState <- "3"

reflectingbdy <- new("markovchain", transitionMatrix=transMatrix,
                 states=stateNames, name="ReflectingBdy")
print(reflectingbdy)

print( summary(reflectingbdy) )

pathLength <- 10 
path <- rmarkovchain(n=pathLength,
                            object = reflectingbdy, t0 = startState)
cat("A sample starting from ", startState, path, "\n")
path <- rmarkovchain(n=pathLength,
                            object = reflectingbdy, t0 = startState)
cat("Another sample starting from ", startState, path, "\n")

if ( length( transientStates( reflectingbdy) ) == 0 ) {
    largeN <- 1000; startStable <- 100
    path <- rmarkovchain(n=largeN,
                         object = reflectingbdy, t0 = startState)
    stablePattern <-  path[startStable:largeN]
    empiricalStable <- rep(0, N+1)
    for (i in seq_along(stateNames)) { 
        empiricalStable[i] <-  mean( stablePattern == (i-1) )
    }
    theorStable <- steadyStates(reflectingbdy)
    cat("Empirical stable distribution: ",  empiricalStable, "\n")
    cat("Theoretical stable distribution: ", theorStable, "\n")
}
\end{lstlisting}

\link{http://www.math.unl.edu/~sdunbar1/Probability/Lessons/MarkovChains/StandardExamples/walkcircle.R}{R script for
  Random Walk on a Circle.}

\begin{lstlisting}[language=R]
library(markovchain)

N <- 7
stateNames <- as.character( 0:N )
transMatrix <- matrix(0, N+1,N+1)
transMatrix[1,2] <- 1/2; transMatrix[1,N+1] <- 1/2
transMatrix[N+1, N] <- 1/2; transMatrix[N+1,1] <- 1/2
for (row in 2:N) {  #row 2 is state 1, row 7 is state 6
    transMatrix[row, row-1] <- 1/2
    transMatrix[row, row+1] <- 1/2
}
startState <- "3"

walkcircle <- new("markovchain", transitionMatrix=transMatrix,
                 states=stateNames, name="Walkcircle")
print(walkcircle)

print( summary(walkcircle) )

pathLength <- 10 
path <- rmarkovchain(n=pathLength,
                            object = walkcircle, t0 = startState)
cat("A sample starting from ", startState, path, "\n")
path <- rmarkovchain(n=pathLength,
                            object = walkcircle, t0 = startState)
cat("Another sample starting from ", startState, path, "\n")

if ( length( transientStates(walkcircle) ) == 0 ) {
    largeN <- 1000; startStable <- 100
    path <- rmarkovchain(n=largeN,
                         object = walkcircle, t0 = startState)
    stablePattern <-  path[startStable:largeN]
    empiricalStable <- rep(0, N+1)
    for (i in seq_along(stateNames)) { 
        empiricalStable[i] <-  mean( stablePattern == (i-1) )
    }
    theorStable <- steadyStates(walkcircle)
    cat("Empirical stable distribution: ",  empiricalStable, "\n")
    cat("Theoretical stable distribution: ", theorStable, "\n")
}
\end{lstlisting}


\link{http://www.math.unl.edu/~sdunbar1/Probability/Lessons/MarkovChains/StandardExamples/ehrenfest1.R}{R script for
   Ehrenfest Urn Model.}

\begin{lstlisting}[language=R]
library(markovchain)

N <- 7
stateNames <- as.character( 0:N )
transMatrix <- matrix(0, N+1,N+1)
transMatrix[1,2] <- 1
transMatrix[N+1, N] <- 1
for (row in 2:N) {  #row 2 is state 1, row 7 is state 6
    transMatrix[row, row-1] <- (row-1)/N
    transMatrix[row,row+1] <- (N-(row-1))/N
}
startState <- "3"

ehrenfest1 <- new("markovchain", transitionMatrix=transMatrix,
                 states=stateNames, name="Ehrenfest1")
print(ehrenfest1)

print( summary(ehrenfest1) )

pathLength <- 10 
path <- rmarkovchain(n=pathLength,
                            object = ehrenfest1, t0 = startState)
cat("A sample starting from ", startState, path, "\n")
path <- rmarkovchain(n=pathLength,
                            object = ehrenfest1, t0 = startState)
cat("Another sample starting from ", startState, path, "\n")

if ( length( transientStates(ehrenfest1) ) == 0 ) {
    largeN <- 1000; startStable <- 100
    path <- rmarkovchain(n=largeN,
                         object = ehrenfest1, t0 = startState)
    stablePattern <-  path[startStable:largeN]
    empiricalStable <- rep(0, N+1)
    for (i in seq_along(stateNames)) { 
        empiricalStable[i] <-  mean( stablePattern == stateNames[i] )
    }
    theorStable <- steadyStates(ehrenfest1)
    cat("Empirical stable distribution: ",  empiricalStable, "\n")
    cat("Theoretical stable distribution: ", theorStable, "\n")
}

  
\end{lstlisting}

\link{http://www.math.unl.edu/~sdunbar1/Probability/Lessons/MarkovChains/StandardExamples/ehrenfest2.R}{R script for
  alternative Ehrenfest Urn Model.}

\begin{lstlisting}[language=R]
library(markovchain)

N <- 7; p <- 1/2; q <- 1 - p;
stateNames <- as.character( 0:N )
transMatrix <- matrix(0, N+1,N+1)
transMatrix[1,1] <- q
transMatrix[1,2] <- p
transMatrix[N+1, N  ] <- q
transMatrix[N+1, N+1] <- p
for (row in 2:N) {
    transMatrix[row, row-1] <- ((N-(row-1))/N)*p
    transMatrix[row,row] <- ((N-(row-1))/N)*q + ((row-1)/N)*p
    transMatrix[row,row+1] <- ((row-1)/N)*q
}
startState <- "3"

ehrenfest2 <- new("markovchain", transitionMatrix=transMatrix,
                 states=stateNames, name="Ehrenfest2")
print(ehrenfest2)

print( summary(ehrenfest2) )

pathLength <- 10 
path <- rmarkovchain(n=pathLength,
                            object = ehrenfest2, t0 = startState)
cat("A sample starting from startState ", startState, path, "\n")
path <- rmarkovchain(n=pathLength,
                            object = ehrenfest2, t0 = startState)
cat("Another sample starting from ", startState, path, "\n")

if ( length( transientStates(ehrenfest2) ) == 0 ) {
    largeN <- 1000; startStable <- 100
    path <- rmarkovchain(n=largeN,
                         object = ehrenfest2, t0 = startState)
    stablePattern <-  path[startStable:largeN]
    empiricalStable <- rep(0, N+1)
    for (i in stateNames) { 
        empiricalStable[i] <-  mean( stablePattern == stateNames[i] )
    }
    theorStable <- steadyStates(ehrenfest2)
    cat("Empirical stable distribution: ",  empiricalStable, "\n")
    cat("Theoretical stable distribution: ", theorStable, "\n")
}


  
\end{lstlisting}

\link{http://www.math.unl.edu/~sdunbar1/Probability/Lessons/MarkovChains/StandardExamples/ehrenfest3.R}{R script for
  another alternative Ehrenfest Urn Model.}

\begin{lstlisting}[language=R]
library(markovchain)

N <- 7; p <- 1/2; q <- 1 - p;
stateNames <- as.character( 0:N )
transMatrix <- matrix(0, N+1,N+1)
transMatrix[1,1] <- 1
transMatrix[N+1, N+1] <- 1
for (row in 2:N) {
    transMatrix[row, row-1] <- ((N-(row-1))/N)*p
    transMatrix[row,row] <- ((N-(row-1))/N)*q + ((row-1)/N)*p
    transMatrix[row,row+1] <- ((row-1)/N)*q
}
startState <- "3"

ehrenfest3 <- new("markovchain", transitionMatrix=transMatrix,
                 states=stateNames, name="Ehrenfest3")
print(ehrenfest3)

print( summary(ehrenfest3) )

pathLength <- 10 
path <- rmarkovchain(n=pathLength,
                            object = ehrenfest3, t0 = startState)
cat("A sample starting from ", startState, path, "\n")
path <- rmarkovchain(n=pathLength,
                            object = ehrenfest3, t0 = startState)
cat("Another sample starting from ", startState, path, "\n")

if ( length( transientStates(ehrenfest3) ) == 0 ) {
    largeN <- 1000; startStable <- 100
    path <- rmarkovchain(n=largeN,
                         object = ehrenfest3, t0 = startState)
    stablePattern <-  path[startStable:largeN]
    empiricalStable <- rep(0, N+1)
    for (i in stateNames) { 
        empiricalStable[i] <-  mean( stablePattern == i )
    }
    theorStable <- steadyStates(ehrenfest3)
    cat("Empirical stable distribution: ",  empiricalStable, "\n")
    cat("Theoretical stable distribution: ", theorStable, "\n")
}


  
\end{lstlisting}

\link{http://www.math.unl.edu/~sdunbar1/Probability/Lessons/MarkovChains/StandardExamples/ehrenfest4.R}{R script for
   yet another alternative Ehrenfest Urn Model.}

\begin{lstlisting}[language=R]
library(markovchain)

N <- 7
stateNames <- as.character( 0:N )
transMatrix <- matrix(0, N+1,N+1)
transMatrix[1,1] <- 1
transMatrix[N+1, N+1] <- 1
for (row in 2:N) {  #row 2 is state 1, row 7 is state 6
    transMatrix[row, row-1] <- (row-1)*(N-(row-1))/N^2
    transMatrix[row,row] <- (row-1)^2/N^2 + (N-(row-1))^2/N^2
    transMatrix[row,row+1] <- (row-1)*(N-(row-1))/N^2
}
startState <- "3"

ehrenfest4 <- new("markovchain", transitionMatrix=transMatrix,
                 states=stateNames, name="Ehrenfest4")
print(ehrenfest4)

print( summary(ehrenfest4) )

pathLength <- 10 
path <- rmarkovchain(n=pathLength,
                     object = ehrenfest4, t0 = startState)
cat("A sample starting from ", startState, ":", path, "\n")
path <- rmarkovchain(n=pathLength,
                     object = ehrenfest4, t0 = startState)
cat("Another sample starting from ", startState, ":", path, "\n")

if ( length( transientStates(ehrenfest4) ) == 0 ) {
    largeN <- 1000; startStable <- 100
    path <- rmarkovchain(n=largeN,
                         object = ehrenfest4, t0 = startState)
    stablePattern <-  path[startStable:largeN]
    empiricalStable <- rep(0, N+1)
    for (i in seq_along(stateNames)) { 
        empiricalStable[i] <-  mean( stablePattern == stateNames[i] )
    }
    theorStable <- steadyStates(ehrenfest4)
    cat("Empirical long-time distribution: ",  empiricalStable, "\n")
    cat("Theoretical stable distribution: ", theorStable, "\n")
}


  
\end{lstlisting}

% \item[Octave]

% \link{http://www.math.unl.edu/~sdunbar1/    .m}{Octave script for .}

% \begin{lstlisting}[language=Octave]

% \end{lstlisting}

% \item[Perl] 

% \link{http://www.math.unl.edu/~sdunbar1/    .pl}{Perl PDL script for .}

% \begin{lstlisting}[language=Perl]

% \end{lstlisting}

% \item[SciPy] 

% \link{http://www.math.unl.edu/~sdunbar1/    .py}{Scientific Python script for .}

% \begin{lstlisting}[language=Python]

% \end{lstlisting}

\end{description}



%%% Local Variables:
%%% mode: latex
%%% TeX-master: t
%%% End:
